% Add command to display corryvreckan/allpix squared
\DeclareMathAlphabet\mathbfcal{OMS}{cmsy}{b}{n}
\newcommand{\corry}{Corryvreckan\xspace}
\newcommand{\corrybold}{\textbf{Corryvreckan}\xspace}

\newcommand{\apsq}{Allpix\textsuperscript{2}\xspace}
\newcommand{\apsqbold}{\textbf{Allpix\textsuperscript{2}}\xspace}

%Define objects
\newcommand{\track}{\texttt{Track}\xspace}
\newcommand{\cluster}{\texttt{Cluster}\xspace}
\newcommand{\pixel}{\texttt{Pixel}\xspace}



% Temporary TODO commands
% \newcommand{\comment}[1]{#1} % DRAFT
\newcommand{\comment}[1]{} % FINAL

\newcommand{\needcite}{\comment{[CITE?] }}
\newcommand{\needref}{\comment{[REF?] }}
\newcommand{\todo}[1]{\comment{[TODO: #1] }}

% Allow compilation with recent pandoc versions:
\newcommand{\passthrough}[1]{\lstset{mathescape=false}#1\lstset{mathescape=true}}

\newcommand{\wip}{\textit{This section is not written yet.}}

% Paragraph with new line
\newcommand{\nlparagraph}[1]{\paragraph{#1}\mbox{}\\}

% Typeset framework parameter and escape underscores:
\DeclareUrlCommand\parameter{\bfseries\urlstyle{tt}}
\newcommand{\command}[1]{\parameter{#1}}

\newcommand{\CPP}{C\nolinebreak[4]\hspace{-.05em}\raisebox{.2ex}{\relsize{-1}{\textbf{++}}}\xspace}

% Typeset directory and file names
\DeclareUrlCommand\dir{\urlstyle{tt}}
\newcommand{\file}[1]{\dir{#1}}

% Typeset module
\DeclareUrlCommand\dir{\urlstyle{tt}}
\newcommand{\module}[1]{\dir{[#1]}}

% Define ini format used in the converted Markdown files
\lstdefinelanguage{Ini}
{
    basicstyle=\ttfamily\small,
    columns=fullflexible,
    morecomment=[s][\color{blue}\bfseries]{[}{]},
    morecomment=[l]{\#},
    morecomment=[l]{;},
    commentstyle=\color{gray}\ttfamily,
    alsoletter={=},
    morekeywords={=},
    otherkeywords={},
    keywordstyle={\color{green}\bfseries}
}

% Warning box
\newsavebox{\warningbox}
\newenvironment{warning}
  {\newcommand\colboxcolor{pink}%
   \begin{lrbox}{\warningbox}%
   \begin{minipage}{\dimexpr\linewidth-2em\relax}}
  {\end{minipage}\end{lrbox}%
   \begin{center}
     \setlength\fboxsep{0pt}
     \colorbox{\colboxcolor}{\setlength\fboxsep{1em}\fbox{\usebox{\warningbox}}}
  \end{center}}

% Command to add all modules
\newcommand{\includemodulesmd}{\def\temp{@CORRYVRECKAN_MODULE_FILES@}\ifx\temp\empty
  \textit{Module documentation not added because Markdown to \LaTeX~conversion was not possible. Pandoc is required for the conversion.}
\else
  \foreach \n in @CORRYVRECKAN_MODULE_FILES@ {\input{\n}}
\fi}

% Command to add a single converted markdown file
\newcommand{\inputmd}[1]{\def\temp{@other_tex_files@}\ifx\temp\empty
  \textit{This section is missing Markdown to \LaTeX~conversion was not possible. Pandoc is required for the conversion.}
\else
  \input{md/#1}
\fi}

% Set bibliography
\addbibresource{usermanual/references.bib}

% Set version
\newcommand{\version}{\lstinline|@CORRYVRECKAN_VERSION@|}
\newcommand{\project}{@CMAKE_PROJECT_NAME@}

% Create addreferences command (overwritten for HTML in config)
\newcommand{\addreferencesline}{\addcontentsline{toc}{chapter}{References}}

% Command to add the license (overwritten for HTML in config)
\newcommand{\addlicense}{
\begin{table}[H]
\centering
\renewcommand{\arraystretch}{1.5}% Spread rows out...
\begin{tabular}{>{\centering\arraybackslash}m{.10\textwidth}>{\raggedright\arraybackslash}m{.90\textwidth}}
 \Large{\ccLogo \ccAttribution} & \footnotesize{This manual is licensed under the Creative Commons Attribution 4.0 International License.\newline To view a copy of this license, visit \url{http://creativecommons.org/licenses/by/4.0/}.} \\
\end{tabular}
\end{table}
}

% Use new lines in FAQ (fixed for HTML in config)
\setlist[description]{style=nextline}
